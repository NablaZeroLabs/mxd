\documentclass[fleqn]{article}

\usepackage[margin=1in]{geometry}
\usepackage{amsmath}

\newcommand{\mxd}{\texttt{mxd}}

\title{\mxd\\Basic Concepts}
\author{F.~Ayala and D.~Rockenzahn and J.~Arrieta}
\date{Dec 08, 2018\\\small{Last Update:~\today}}

\begin{document}
\maketitle


\section{Orbital Elements}
In this section we explain some orbital element sets and how to convert to and
from different sets.
\subsection{Cartesian Elements}
The Cartesian elements are
\begin{align}
  \vec{r}(t), \vec{v}(t).
\end{align}
They define position and velocity at a given time.

\subsection{Keplerian Elements}
The Keplerian elements are
\begin{align}
  a, e, i, \omega, \Omega, f(t).
\end{align}
They define semimajor axis, eccentricity, inclination, right ascension of the
ascending node, argument of periapsis, and true anomaly. Notice that only true
anomaly varies with time.

In order for the previously mentioned elements to correctly describe an orbit, it must meet two conditions: The orbit must be osculating, and elliptical.

\paragraph{Osculating Orbits}
An orbit is said to be osculating if the orbiting body has only one attractor, without the preturbation of any other gravitational force force, and thus follows a kepler orbit.

\paragraph{Elliptical orbits}  
TODO: Explain why the given elements only apply to elliptical (non-circular,
non-parabolic, non-hyperbolic) orbits.

\subsection{Cartesian-Keplerian Conversion}
TODO: How to get from Cartesian to Keplerian elements.

\subsection{Keplerian-Cartesian Conversion}
TODO: How to get from Keplerian to Cartesian elements.

\section{Basic Vector Algebra}
TODO: Add basic vector algebra

\section{Drawing Conic Sections}
TODO: Explain how would one go about drawing a conic section (which is in a
plane) in some three-dimensional frame.

\end{document}
